% !TEX root=main.tex
\section{Introduction}
~\label{sec:introduction}
\hspace{8pt}
In recent decades, the attention on \acrfull{iaq} has become a focal point for scientific researchers, political institutions, and environmental regulators worldwide. This heightened interest aims to improve the living conditions, health, and well-being of individuals residing in buildings. Numerous research efforts have indicated both qualitative and quantitative changes in \acrshort{iaq} over time, emphasizing an increase in pollutants and their concentrations. An estimation showed that individuals spend approximately 90\% of their time in various indoor environments, including homes, gyms, schools, workplaces, and vehicles. Consequently, \acrshort{iaq} significantly influences health and overall quality of life. For many people, the health hazards associated with indoor air pollution may surpass those linked to outdoor pollution. Specifically, poor \acrshort{iaq} can pose a significant risk to vulnerable demographics such as children, young adults, the elderly, and those with chronic respiratory or cardiovascular conditions~\cite{cincinelli_2017}. This is why monitoring the \acrshort{iaq} is important. \\

\acrfull{daq} System samples signals containing real-world physical conditions--like voltage or current--and converts them into digital signals that can be handled for another device. Many signals that are sampled from sensors and transducers must be conditioned before they can be transformed into digital signals~\cite{di_paolo_2013}. A signal conditioning circuit is designed to convert a signal from the sensing element up to a format that can be processed by the load device, usually an \acrfull{adc}. For effective operation, a signal conditioner must dutifully serve two entities: the sensor and the load device. The input characteristics of the signal conditioner should match the output characteristics of the sensor, while its output should produce a voltage that can easily be interfaced with an \acrshort{adc} or another load device~\cite{fraden_2016}. Developing a \acrshort{daq} System that can monitor the \acrshort{iaq} is the main goal of this work. \\

Arduino UNO Rev 3 is a microcontroller board based on the ATmega328P that has 14 digital input/output pins (of which 6 can be used as \acrfull{pwm} outputs), 6 analog inputs, a 16 MHz ceramic resonator, a \acrfull{usb} connection, a power jack, an \acrfull{icsp} header, and a reset button~\cite{arduino_2024}. The ATmega328P datasheet~\cite{atmega328p} indicates it contains a successive approximation \acrshort{adc} with a resolution of 10-bit. The \acrshort{adc} is connected to an 8-channel analog multiplexer which allows eight single-ended voltage inputs constructed from the pins of Port A~\cite{atmega328p}.
