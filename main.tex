%----------------------------------------------------------------------------------------
%	PACKAGES AND DOCUMENT CONFIGURATIONS
%----------------------------------------------------------------------------------------
% !TEX root=main.tex
\documentclass[10pt,a4paper]{article}

\addtolength{\oddsidemargin}{-.5in}
\addtolength{\evensidemargin}{-.5in}
\addtolength{\textwidth}{1in}
\addtolength{\topmargin}{-.5in}
\addtolength{\textheight}{1in}

% Graphics
\usepackage{caption}
\usepackage{subcaption}
\usepackage{graphicx}
\graphicspath{{figures/}}

% Math
\usepackage{amssymb}
\usepackage{amsmath}

% Other
\usepackage{algorithmic}
\usepackage{array}
\usepackage{lipsum}
\usepackage{hyperref}
\usepackage[acronym]{glossaries}
\usepackage{url}
\usepackage{float}
\usepackage{listings}
\usepackage{xcolor}

\definecolor{codegreen}{rgb}{0,0.6,0}
\definecolor{codegray}{rgb}{0.5,0.5,0.5}
\definecolor{codepurple}{rgb}{0.58,0,0.82}
\definecolor{backcolour}{rgb}{0.95,0.95,0.92}

\lstdefinestyle{mystyle}{
    backgroundcolor=\color{backcolour},   
    commentstyle=\color{codegreen},
    keywordstyle=\color{magenta},
    numberstyle=\tiny\color{codegray},
    stringstyle=\color{codepurple},
    basicstyle=\ttfamily\footnotesize,
    breakatwhitespace=false,         
    breaklines=true,                 
    captionpos=b,                    
    keepspaces=true,                 
    numbers=left,                    
    numbersep=5pt,                  
    showspaces=false,                
    showstringspaces=false,
    showtabs=false,                  
    tabsize=2
}

\lstset{style=mystyle}

\setlength{\parindent}{12pt}

\providecommand{\keywords}[1]
{
  \small	
  \textbf{\textit{Keywords---}} #1
}

%----------------------------------------------------------------------------------------
% Acronyms
%----------------------------------------------------------------------------------------
\newacronym{adc}{ADC}{Analog-Digital Converter}
\newacronym{daq}{DAQ}{Data Acquisition}
\newacronym{dc}{DC}{Direct Current}
\newacronym{diy}{DIY}{Do It Yourself}
\newacronym{i2c}{I2C}{Inter-Integrated Circuit}
\newacronym{iaq}{IAQ}{Indoor Air Quality}
\newacronym{icsp}{ICSP}{In-Circuit Serial Programming}
\newacronym{iir}{IIR}{Infinite Impulse Response}
\newacronym{lfn}{LFN}{Low-Frequency Noise}
\newacronym{lpg}{LPG}{Liquefied Petroleum Gas}
\newacronym{pcb}{PCB}{Printed Circuit Board}
\newacronym{psd}{PSD}{Power Spectral Density}
\newacronym{pwm}{PWM}{Pulse-Width Modulation}
\newacronym{scl}{SCL}{Serial Clock}
\newacronym{sda}{SDA}{Serial Data}
\newacronym{spi}{SPI}{Serial Peripheral Interface}
\newacronym{usb}{USB}{Universal Serial Bus}
\newacronym{twi}{TWI}{Two Wire Interface}


%----------------------------------------------------------------------------------------
% MAIN PART
%----------------------------------------------------------------------------------------
\begin{document}

\title{Data Acquisition System for Indoor Air Quality Monitoring using Arduino UNO R3 Board}
\author{Erick Andrés Obregón Fonseca}
\date{\today}
\maketitle




%----------------------------------------------------------------------------------------
% Abstract
%----------------------------------------------------------------------------------------
\begin{abstract}
\normalsize
In recent years, scientists, politicians, and environmental institutions have focused on Indoor Air Quality. Later research has demonstrated that poor air quality can lead to different respiratory problems and cardiovascular diseases. This project aims to develop a prototype for an Indoor Air Quality Monitoring System using an Arduino UNO board. For such a task, variables like temperature, humidity, and gas concentrations are measured to detect any condition that can cause long-term health issues. First, some basic concepts are introduced. Then, the following methodology is described, where the development of the system is shown. This includes the choice of sensor, design of amplification, and filtering stages for analog sensor, and the proposed algorithm for data sampling. Finally, the results and conclusions are discussed regarding the design of the conditioning and filtering stage of the analog sensor.
\end{abstract}

%----------------------------------------------------------------------------------------
% Keywords
%----------------------------------------------------------------------------------------
\keywords{Arduino UNO, DAQ system, I2C, IQA, sensors}

%----------------------------------------------------------------------------------------
% Table of Content
%----------------------------------------------------------------------------------------
\setcounter{tocdepth}{2}
\tableofcontents


\clearpage

%----------------------------------------------------------------------------------------
% Main Part
%----------------------------------------------------------------------------------------
% !TEX root=main.tex
\section{Introduction}
\label{sec:introduction}
\hspace{8pt}
In recent decades, the attention on \acrfull{iaq} has become a focal point for scientific researchers, political institutions, and environmental regulators worldwide. This heightened interest aims to improve the living conditions, health, and well-being of individuals residing in buildings. Numerous research efforts have indicated both qualitative and quantitative changes in \acrshort{iaq} over time, emphasizing an increase in pollutants and their concentrations. An estimation showed that individuals spend approximately 90\% of their time in various indoor environments, including homes, gyms, schools, workplaces, and vehicles. Consequently, \acrshort{iaq} significantly influences health and overall quality of life. For many people, the health hazards associated with indoor air pollution may surpass those linked to outdoor pollution. Specifically, poor \acrshort{iaq} can pose a significant risk to vulnerable demographics such as children, young adults, the elderly, and those with chronic respiratory or cardiovascular conditions \cite{cincinelli_2017}. This is why monitoring the \acrshort{iaq} is important. \\

\acrfull{daq} System samples signals containing real-world physical conditions--like voltage or current--and converts them into digital signals that can be handled for another device. Many signals that are sampled from sensors and transducers must be conditioned before they can be transformed into digital signals \cite{di_paolo_2013}. A signal conditioning circuit is designed to convert a signal from the sensing element up to a format that can be processed by the load device, usually an \acrfull{adc}. For effective operation, a signal conditioner must dutifully serve two entities: the sensor and the load device. The input characteristics of the signal conditioner should match the output characteristics of the sensor, while its output should produce a voltage that can easily be interfaced with an \acrshort{adc} or another load device \cite{fraden_2016}. Developing a \acrshort{daq} System that can monitor the \acrshort{iaq} is the main goal of this work. \\

Arduino UNO Rev 3 is a microcontroller board based on the ATmega328P that has 14 digital input/output pins (of which 6 can be used as \acrfull{pwm} outputs), 6 analog inputs, a 16 MHz ceramic resonator, a \acrfull{usb} connection, a power jack, an \acrfull{icsp} header, and a reset button \cite{arduino_2024}. The ATmega328P datasheet \cite{atmega328p} indicates it contains a successive approximation \acrshort{adc} with a resolution of 10-bit. The \acrshort{adc} is connected to an 8-channel analog multiplexer which allows eight single-ended voltage inputs constructed from the pins of Port A \cite{atmega328p}.

\section{Methodology}
\label{sec:methodology}


\subsection{Subsection}
\label{sec:methodology:subsection}

\section{Results and Discusion}
\label{sec:results_and_discusion}


\subsection{Subsection}
\label{sec:results_and_discusion:subsection1}


\subsection{Subsection}
\label{sec:results_and_discusion:subsection2}


\section{Conclusion}
\label{sec:conclusion}
\hspace{8pt}
In this report, I presented an IAQ monitor which is important for keeping people's health and notifying them when the air quality can be long-term harmful. \\

\acrshort{daq} systems are designed to collect, measure, and process data from various sensors, including analog, digital, and other connected for any data transmission protocol like SPI and I2C. They play an important role in monitoring and controlling systems with different applications. Signals acquired from sensors are vulnerable to different kinds of noise that can affect the measurement and lead to errors. It is also important to condition the analog signal to meet the ADC requirements. \\

Filtering the signals helps to clean the possible noise that they can bring to the system. Bessel filters provide a maximally flat response with minimal ripple, unlike other filters like Butterworth or Chebyshev. This aided by its characteristics of preserving phase accuracy and minimizing signal quality loss, makes it ideal for applications where these features are critical. \\

The operational amplifiers have different parameters that need to be taken into consideration to choose the right model for the application. Low power consumption is important in those systems that are battery-based and will be stand-alone for long times. Low offset voltage helps to improve the precision of the output, by minimizing the voltage difference between its terminal when output is zero. \\

Designing a \acrshort{daq} system can be time-intensive. Using tools like Texas Instruments Filter Design Tool can help to minimize the time of the design development and reduce the human error during value calculation.




\clearpage

%----------------------------------------------------------------------------------------
% Bibliography
%----------------------------------------------------------------------------------------
\clearpage
\addcontentsline{toc}{section}{References}
\bibliography{bibliography/sample}{}
\bibliographystyle{ieeetr}

\end{document}
